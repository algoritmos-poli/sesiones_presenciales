% --- PREÁMBULO ---
% Define el tipo de documento. 'article' es ideal para documentos cortos.
\documentclass[12pt, letterpaper]{article}

% --- PAQUETES ESENCIALES ---
% Soporte para caracteres en español (acentos, ñ)
\usepackage[utf8]{inputenc}

% Configuración del idioma español para la correcta separación de sílabas y títulos.
\usepackage[spanish]{babel}

% Paquete para un manejo avanzado de las matemáticas.
\usepackage{amsmath}

% Paquete para configurar los márgenes del documento.
\usepackage[margin=2.5cm]{geometry}

% --- INFORMACIÓN DEL DOCUMENTO ---
\title{Mi Documento de Prueba en \LaTeX}
\author{Mi Nombre}
\date{\today} % Usa la fecha actual

% --- INICIO DEL DOCUMENTO ---
\begin{document}

% Crea la portada con la información de arriba
\maketitle

% --- CONTENIDO ---
\section{Introducción a la Prueba}

Este es un documento de prueba para verificar que mi entorno \LaTeX{} funciona correctamente. Si puedes leer esto, incluyendo la letra **ñ** y las palabras con **acentuación**, significa que los paquetes `inputenc` y `babel` están funcionando bien.

\subsection{Elementos a Verificar}

A continuación, se listan algunos elementos comunes en un documento:
\begin{itemize}
    \item Listas como esta.
    \item Fórmulas matemáticas.
    \item Referencias cruzadas.
\end{itemize}

\section{Prueba de Matemáticas y Referencias}

Las matemáticas son una de las fortalezas de \LaTeX. A continuación, se muestra la famosa ecuación de la relatividad de Einstein, que será numerada automáticamente:

\begin{equation}
E = mc^2
\label{eq:einstein}
\end{equation}

El propósito de `latexmk` es compilar el documento las veces que sea necesario para que las referencias funcionen. Por ejemplo, como vemos en la ecuación \ref{eq:einstein}, la energía es igual a la masa por la velocidad de la luz al cuadrado.

Si el número de la ecuación aparece correctamente en la frase anterior (un "1"), ¡la prueba ha sido un éxito!

\end{document}
% --- FIN DEL DOCUMENTO ---